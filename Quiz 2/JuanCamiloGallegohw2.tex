\documentclass{article}

\title{WC(1)                         User Commands                        WC(1) wc - print newline, word, and byte counts for each file}
\author{Paul Rubi}
\date{January 2015}

\begin{document}
\maketitle
\section{astract}
  wc [OPTION]... [FILE]...
       wc [OPTION]... --files0-from=F

\section{DESCRIPTION}
 Print newline, word, and byte counts for each FILE, and a total line if
       more than one FILE is specified.  With no FILE, or when FILE is -, read
       standard  input.   A  word  is a non-zero-length sequence of characters
       delimited by white space.  The options below  may  be  used  to  select
       which counts are printed, always in the following order: newline, word,
       character, byte, maximum line length. -c, --bytes
              print the byte counts

       -m, --chars
              print the character counts

       -l, --lines
              print the newline counts

       --files0-from=F
              read input from the files specified by NUL-terminated  names  in
              file F; If F is - then read names from standard input

       -L, --max-line-length
              print the length of the longest line

       -w, --words
              print the word counts

       --help display this help and exit

       --version
              output version information and exit
\section{AUTHOR}
 Written by Paul Rubin and David MacKenzie
\section{REPORT BUGS}
 Report wc bugs to bug-coreutils@gnu.org
       GNU coreutils home page: <http://www.gnu.org/software/coreutils/>
       General help using GNU software: <http://www.gnu.org/gethelp/>
       Report wc translation bugs to <http://translationproject.org/team/>
\section{COPYRIGHT}
 Copyright  ©  2013  Free Software Foundation, Inc.  License GPLv3+: GNU
       GPL version 3 or later <http://gnu.org/licenses/gpl.html>.
       This is free software: you are free  to  change  and  redistribute  it.
       There is NO WARRANTY, to the extent permitted by law.
\section{SEE ALSO}
 The  full  documentation  for wc is maintained as a Texinfo manual.  If
       the info and wc programs are properly installed at your site, the  com‐
       mand

              info coreutils 'wc invocation'

       should give you access to the complete manual.
\end{document}
